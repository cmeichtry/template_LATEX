% Uso de subfiles.tex
% Copyright (C) 2024 Cristian Meichtry
% https://github.com/cmeichtry/template_LATEX

\cursiva{\negrita{\subrayar{\LARGE{Uso de subfiles}}}}

El uso de subfiles permite compilar cada sección por separado para evitar demoras excesivas si se tiene un archivo extenso.
Si se compila desde dentro de una sección, solo será procesada esa parte del archivo. Para unificar todo y obtener el PDF completo la compilación debe ser realizada desde main.tex

En caso de querer sacar esta opción los pasos a seguir son:
\begin{enumerate}
    \item Entrar a cada seeción y eliminar las lineas:
        \begin{itemize}
            \item \texttt{documentclass[../main.tex]{subfiles}}
            \item \texttt{begin{document}}
            \item \texttt{end{document}}
        \end{itemize}
    \item Ir a \comilladoble{main.tex} y cambiar todos los \comilladoble{subfiles} por \comilladoble{input}
\end{enumerate}

