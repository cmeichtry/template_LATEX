En este .tex hay información que puede ser de ayuda para insertar elementos y para escribir en modo matemático. En caso de necesitar más ayuda -> google.com

%%%% REFERENCIAS %%%%
%Citar con \autoref{fig:test} las figuras, \autoref{tab:test} las tablas, \eqref{eq:test} las ecuaciones

%%%% IMAGENES %%%%
%\begin{figure}[H]
%  \centering
%   \includegraphics[height = 8cm]{imagenes/seccionN/help.jpg}
%   \caption{%TITULO DE LA IMAGEN}
%    \label{fig:test}
%\end{figure}

%TABLA
%\begin{table}[htbp]
%  \centering
%  \resizebox{10cm}{!}{
%    \begin{tabular}{|c|c|c|c|c|c|c|}
%    \hline
%    a  & a & a & a & a & a\\
%    \hline
%    a  & a & a & a & a & a\\
%    \hline
%    a  & a & a & a & a & a\\
%    \hline
%    \end{tabular}
%    }
%    \caption{Corriente de Bias y tensión de offset de LF356 y TL081.}
%  \label{tab:test}
%\end{table}
%

%ECUACIONES
%\begin{equation}
%    2 + 2 = 5
%     \label{eq:test}
%\end{equation}

%ECUACIONES DE REFERENCIA
\begin{equation*}
	a\cdot b
\end{equation*}

\begin{equation*}
	\frac{a}{b}
\end{equation*}

\begin{equation*}
	\int_a^b f(x)\, dx. 
\end{equation*}

\begin{equation*}
	\sum_{i=1}^n f(x_i) \Delta x
\end{equation*}

\begin{equation*}
	\lim_{\Delta \to 0} \frac{f(x_0+\Delta x)-f(x_0)}{\Delta x}
\end{equation*}

\begin{equation*}
	\lvert -5 \rvert=5
\end{equation*}

\begin{equation*}
	\sqrt{x+1}
\end{equation*}

\begin{equation*}
	\vec{a} =\begin{pmatrix} 1\\2\\3 \end{pmatrix} \text{, } \widehat{\vec{a}} \text{ o } \overrightarrow{AB}
\end{equation*}

\begin{equation*}
	f(x) = \begin{cases}
	x^2 & \text{si } x>2,\\
	x-1 & \text{si } x\leq 2
	\end{cases}
\end{equation*}

\begin{equation*}
	\left(\frac{a}{b} \right)
\end{equation*}

\begin{equation*}
	\pm \quad \infty \quad \leq \quad \geq \quad \circ \quad \in \quad \notin \quad \neq \quad \bullet \quad \Leftrightarrow \quad \Updownarrow \quad \times \quad \angle
\end{equation*}

\begin{equation*}
	\sin(x) \quad \cos(x) \quad \tan(x) \quad \ln(x) \quad \log(x) \quad \exp(x)
\end{equation*}

\begin{align*}
	(a+b)^2 &= (a+b)(a+b)\\
	&= a^2+ab+ba+b^2\\
	&=a^2+2ab+b^2
\end{align*}