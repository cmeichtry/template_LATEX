\def \universidad{Instituto Tecnológico de Buenos Aires}
\def \unidad{NUMERO DE TP}
\def \carrera{Departamento de Ingeniería Electrónica}
\def \materia{MATERIA}
\def \titulo{TITULO}

\def \grupo{NUMERO DE GRUPO}

\def \autores{
\noindent
    \textbf{GRUPO N° \grupo}\\
    Integrante 1\\
    Integrante 2\\
    Integrante 3\\
    Integrante 4\\
    Integrante 5\\
}
\def \profesores{
    Ing. Profesor 1\\
    Ing. Profesor 2\\
}

\def \Colaboradores{ %nombres de los ayudantes
    
}

\def \fecha{
    \noindent
    Fecha de entrega: \today
    
    \noindent
    Fecha de corrección:
}

\def \seteocontadores{
\setcounter{equation}{0} %empezar numeración desde 1 para ecuaciones
\setcounter{figure}{0} %empezar numeración desde 1 para figuras
\setcounter{table}{0} %empezar numeración desde 1 para tablas
}

\def \formatoapendice{
\renewcommand{\thetable}{\nombreapendice{}.\arabic{table}}
\renewcommand{\thefigure}{\nombreapendice{}.\arabic{figure}}
\rfoot{\nombreapendice{}.\thepage}
}

%--------------------Funciones para usar en mathmode--------------------%
\newcommand{\derpar}[2]{\frac{\partial #1}{\partial #2}} %derivada parcial
\newcommand{\der}[2]{\frac{d #1}{d #2}} %derivada "normal"
\newcommand{\paren}[1]{\left( #1 \right)} %Parentesis
\newcommand{\corch}[1]{\left[ #1 \right]} %corchetes
\newcommand{\llave}[1]{\left \{ #1 \right \}} %llaves
\newcommand{\supind}[2]{#1^{#2}} %supraindice
\newcommand{\subind}[2]{#1_{#2}} %supraindice
\newcommand{\ohm}{\Omega} %Escribe el simbolo Ω

%--------------------Funciones para formato de texto--------------------%
\newcommand{\cursiva}[1]{\textit{#1}}
\newcommand{\negrita}[1]{\textbf{#1}}
\newcommand{\subrayar}[1]{\underline{#1}}