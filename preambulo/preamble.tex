% preamble.tex
% Copyright (C) 2023 Cristian Meichtry
% https://github.com/cmeichtry/template_LATEX

%--------------------BLOQUE DE FORMATO--------------------%
\usepackage[a4paper,includehead,includefoot, left=2cm,top=1.2cm,right=2cm,bottom=1.5cm,headheight=25pt]{geometry} %margenes.
%--------------------FIN DE BLOQUE DE FORMATO--------------------%

%--------------------CONFIGURACION DE IDIOMA--------------------%
\usepackage[spanish,es-tabla]{babel} % pone el idioma en español, es-tabla para Tabla en lugar de Cuadro
\usepackage[utf8]{inputenc} % Codificación de entrada, carácteres acentuados, ñ
\usepackage[T1]{fontenc} % Codificación de fuente, habilita caracteres no latinos
\usepackage{lmodern} % Fuente compatible
%--------------------FIN DE CONFIGURACION DE IDIOMA--------------------%

\usepackage{amsmath,amsthm,amsfonts,amssymb,amscd}  %para ecuaciones
\usepackage{enumerate} %permite hacer listas numeradas y por niveles
\usepackage{mathrsfs} %letras mayúsculas especiales, cursiva, gotica, etc
\usepackage{dsfont} %letras mayúsculas para conjuntos (Ej: Naturales)
\usepackage{xcolor} %permite escribir texto en color
\usepackage{graphicx} %para insertar graficos e imagenes.
\usepackage{listings} %para ingresar codigos de MATLAB, Python, Fortran, etc
\usepackage{bm} %permite usar simbolos matematicos en negrita
\usepackage{nicefrac} %Permite escribir fracciones en el medio del texto de una forma elegante \nicefrac{a}{b}
\usepackage{booktabs} %modificación de las lineas en las tablas
\usepackage{parskip} %modificación de sangría
\usepackage[center]{caption} %permite modificar el formato de los caption para tablas/figuras
\usepackage{array} %permite escribir en columnas sin ser una tabla, por ejemplo, matrices
\usepackage{multirow} %filas múltiples en tablas
\usepackage{stackrel} %Permite escribir cosas arriba o abajo de otras, \stackrel[lo de abajo] {lo de arriba}{lo del medio}
\usepackage{subfigure} %Permite poner subfiguras
\usepackage{cancel} %permite tachar terminos en ecuaciones
\usepackage{comment} %para comentar grandes partes de codigo, usar \begin{comment}
\usepackage{setspace} %permite cambiar espacio entre lineas
\usepackage{steinmetz} %mejora la forma de escribir numeros complejos
\usepackage{fancyhdr} % Modificar encabezados y pies de paginas.
\usepackage{titlesec} %modifición de secciones
\usepackage{float} %para el float H en tablas.
\usepackage{etoolbox} %crea condicionales, puedo tener dos PDFs distintos desde un solo archivo fuente
\usepackage[colorlinks=false,linkcolor=black]{hyperref} %pone vinculos en el indice y en las referencias
\usepackage{lipsum} %introduce bloques de texto random, usado para tests
%--------------------Titulos y enumeración de secciones--------------------%
%titulos de secciones y subsecciones.
\titleformat{\section}[hang]{\normalfont\bfseries\centering}{\thesection}{5pt.}{} %modificación secciones.
\titleformat{\subsection}[hang]{\normalfont\bfseries\raggedright}{\thesubsection}{5pt}{} %modificación subsecciones.
\titleformat{\subsubsection}[hang]{\normalfont\itshape}{\thesubsubsection}{5pt}{} %modificación subsecciones.
%\titleformat{\subsubsection}[hang]{\itshape}{Section \thesubsubsection}{}{\hspace{1cm} $\bullet$ \hspace{0.1cm}}[] %modificación subsecciones.

%enumeración de secciones y subsecciones.
\renewcommand{\labelenumi}{\arabic{section}.\arabic{enumi}}
\renewcommand{\labelenumii}{\arabic{section}.\arabic{enumi}.\arabic{enumii}}
\renewcommand{\labelenumiii}{\arabic{section}.\arabic{enumi}.\arabic{enumii}.\arabic{enumiii}}
\renewcommand{\labelenumiv}{\arabic{section}.\arabic{enumi}.\arabic{enumii}.\arabic{enumiii}.\arabic{enumiv}}

\renewcommand{\thefigure}{\arabic{section}.\arabic{figure}}  %definición enumeración de ecuaciones.
\renewcommand{\thetable}{\arabic{section}.\arabic{table}} %definición enumeración de tablas.

\setcounter{tocdepth}{2} %hasta que nivel aparece en el índice
\setcounter{secnumdepth}{3} %hasta que nivel se enumera
%niveles 1:section, 2:subsection, 3:subsubsection...