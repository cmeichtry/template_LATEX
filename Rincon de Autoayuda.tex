% Rincon de Autoayuda.tex
% Copyright (C) 2023 Cristian Meichtry
% https://github.com/cmeichtry/template_LATEX

%En este .tex hay información que puede ser de ayuda para insertar elementos y para escribir en modo matemático. En caso de necesitar más ayuda -> google.com

%%%% REFERENCIAS %%%%
%Citar con \autoref{fig:test} las figuras, \autoref{tab:test} las tablas, \eqref{eq:test} las ecuaciones

%%%% IMAGENES %%%%
%\begin{figure}[H]
%  \centering
%   \includegraphics[height = 8cm]{imagenes/seccionN/help.jpg}
%   \caption{%TITULO DE LA IMAGEN}
%    \label{fig:test}
%\end{figure}

%TABLA
%\begin{table}[htbp]
%  \centering
%  \resizebox{10cm}{!}{
%    \begin{tabular}{|c|c|c|c|c|c|c|}
%    \hline
%    a  & a & a & a & a & a\\
%    \hline
%    a  & a & a & a & a & a\\
%    \hline
%    a  & a & a & a & a & a\\
%    \hline
%    \end{tabular}
%    }
%    \caption{Corriente de Bias y tensión de offset de LF356 y TL081.}
%  \label{tab:test}
%\end{table}
%

%ECUACIONES
%\begin{equation}
%    2 + 2 = 5
%     \label{eq:test}
%\end{equation}

%ECUACIONES DE REFERENCIA
%https://manualdelatex.com/simbolos
\centering{\subrayar{Producto}}
\begin{equation*}
	a\cdot b
\end{equation*}

\centering{\subrayar{División}}
\begin{equation*}
	\frac{a}{b}
\end{equation*}

\centering{\subrayar{Integral}}
\begin{align*}
	&\int f(x)\,dx\\
        &\iint f(x,y)\,dx\,dy\\
        &\iiint f(x,y,z)\,dx\,dy\,dz\\
        &\int\limits_{a}^{b}f(x)\,dx\\
        &\int_{a}^{b}f(x)\,dx\\
        &\int_{a}^{b}\int_{c}^{d}f(x,y)\,dx\,dy
\end{align*}

\centering{\subrayar{Sumatoria}}
\begin{equation*}
	\sum_{i=1}^n f(x_i) \Delta x
\end{equation*}

\centering{\subrayar{Productoria}}
\begin{equation*}
	\prod_{i=1}^n f(x_i) \Delta x
\end{equation*}

\centering{\subrayar{Límites}}
\begin{align*}
	&\lim_{x \to 0} \frac{1}{x} = \,\infty \\ 
        &\frac{1}{x}\stackrel[] {x \to 0}{\longrightarrow}\,\infty
\end{align*}

\centering{\subrayar{Módulo}}
\begin{equation*}
	\lvert -5 \rvert=5
\end{equation*}

\centering{\subrayar{Raíces}}
\begin{align*}
	&\sqrt{x+1}\\
        &\sqrt[3]{x+1}\\
        &\sqrt[4]{x+1}
\end{align*}

\centering{\subrayar{Matrices y vectores}}
\begin{align*}\centering
        &    A_{m,n} = 
            \begin{pmatrix}
                a_{1,1} & a_{1,2} & \cdots & a_{1,n} \\
                a_{2,1} & a_{2,2} & \cdots & a_{2,n} \\
                \vdots  & \vdots  & \ddots & \vdots  \\
                a_{m,1} & a_{m,2} & \cdots & a_{m,n} 
            \end{pmatrix}\\
        &\vec{a} =
            \begin{pmatrix} 
                1\\2\\3 
            \end{pmatrix}\\
        &\widehat{\vec{a}}\\
        &\overrightarrow{AB}
\end{align*}

\centering{\subrayar{Sistemas}}
\begin{equation*}
	f(x) = \begin{cases}
	x^2 & \text{si } x>2,\\
	x-1 & \text{si } x\leq 2
	\end{cases}
\end{equation*}

\centering{\subrayar{Simbolos}}
\begin{equation*}
\pm \, \infty \, \leq \, \geq \, \circ \, \in \, \notin \, \subset \, \subseteq \, \cap \, \cup \, A^\complement \, \exists \, \forall \, \neq \, \bullet  \, \times \, \angle \, \land \, \lor \, \neg \, \overline{AB} \, \triangle \, \square
\end{equation*}

\centering{\subrayar{Flechas}}
\begin{equation*}
\leftarrow \, \rightarrow \, \leftrightarrow \, \Leftarrow \, \Rightarrow \, \Leftrightarrow \, \uparrow \, \downarrow \, \Updownarrow
\end{equation*}

\centering{\subrayar{Funciones}}
\begin{equation*}
	\sin(x) \quad \cos(x) \quad \tan(x) \quad \ln(x) \quad \log(x) \quad \exp(x)
\end{equation*}

\centering{\subrayar{Entorno \cursiva{align}}}
%se usa el & para alinear los distintos renglones
\begin{align*}
	(a+b)^2 &= (a+b)(a+b)\\
	&= a^2+ab+ba+b^2\\
	&=a^2+2ab+b^2
\end{align*}

\centering{\subrayar{Distintos tipos de letras}}
\begin{equation*}
    \mathrm{R}, \mathbb{R}, \mathcal{R}, \mathfrak{R}, \mathbf{R}, \mathsf{R}, \mathit{R}, \mathscr{R}
\end{equation*}

\centering{\subrayar{\large{Ejemplos}}}
\begin{equation*}
\mathscr{F}[x(t)](f)=X(f)=\int_{-\infty}^{\infty} x(t) \supind{e}{-i 2\pi f t}\, dt
\end{equation*}

\begin{equation*}
\mathscr{F}[x(n)](f)=X(f)=\sum_{-\infty}^{\infty} x(n) \supind{e}{-i 2\pi f n}
\end{equation*}

\begin{equation*}
I = \iint{\overrightarrow{J}\cdot\overrightarrow{ds}}=J\cdot A
\end{equation*}

\begin{equation*}
\subind{V}{esfera} = \iiint_{Bola}\,dV = \int_{0}^{2\pi}\int_{0}^{\pi}\int_{0}^{R}{r^2\sin(\phi)}\,dr\,d\phi\,d\theta=\frac{4}{3}\pi R^2
\end{equation*}

\begin{equation*}
f'(x)=\lim_{\Delta x \to 0} \frac{f(x+\Delta x)-f(x)}{\Delta x}
\end{equation*}

\begin{equation*}
	sgn(x) = \begin{cases}
	-1 & \text{si } x<0,\\
	0 & \text{si } x=0,\\
        1 & \text{si } x>0
	\end{cases}
\end{equation*}

\begin{equation*}
	\frac{d^2}{dt^2}i(t)+\frac{R}{L}\frac{d}{dt}i(t)+\frac{1}{LC}i(t)=0
\end{equation*}

\begin{equation*}
    \mathbf{X} = 
            \begin{pmatrix}
                1&2\\
                3&4\\
                5&6
            \end{pmatrix}\in \supind{\mathbb{R}}{3\times 2}
\end{equation*}
