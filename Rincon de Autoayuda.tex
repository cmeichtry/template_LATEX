% Rincon de Autoayuda.tex
% Copyright (C) 2023 Cristian Meichtry
% https://github.com/cmeichtry/template_LATEX

\cursiva{\negrita{\subrayar{Rincón de Autoayuda:}} En este archivo hay información básica sobre algunos elementos de \LaTeX\, que puede ser de ayuda para insertar elementos y/o para escribir en modo matemático. En caso de necesitar más herramientas pueden ser útiles sitios como \url{manualdelatex.com}, \url{https://www.overleaf.com/learn} o \url{google.com}}\\


\negrita{\subrayar{REFERENCIAS:}} Referenciar con \autoref{fig:test}, \autoref{subfig:test1} las figuras y subfiguras, \autoref{tab:test} las tablas y con \eqref{eq:test} las ecuaciones

\negrita{\subrayar{IMÁGENES}}
\begin{figure}[H]
  \centering
   \includegraphics[height = 5cm]{imagenes/logo.png}
   \caption{Ejemplo}
   \label{fig:test}
\end{figure}

\begin{figure}[H] %si se quiere una figura al lado de otra la suma de los anchos de cada imagen no debe ser mayor a textwidth, de lo contrario quedan una abajo de la otra
    \centering
    \begin{subfigure}[b]{0.3\textwidth}
        \centering
        \includegraphics[width = \textwidth]{imagenes/logo.png}
        \caption{}
        \label{subfig:test1}
    \end{subfigure}
    \hfill
    \begin{subfigure}[b]{0.3\textwidth}
        \centering
        \includegraphics[width = \textwidth]{imagenes/logoITBA.png}
        \caption{}
        \label{subfig:test2}
    \end{subfigure}
    \hfill
    \begin{subfigure}[b]{0.3\textwidth}
        \centering
        \includegraphics[width = \textwidth]{imagenes/logoITBA.png}
        \caption{}
        \label{subfig:test3}
    \end{subfigure}
        \caption{Ejemplo de subfiguras}
        \label{fig:test1}
\end{figure}

\begin{figure}[H] %si se quiere una figura al lado de otra la suma de los anchos de cada imagen no debe ser mayor a textwidth, de lo contrario quedan una abajo de la otra
    \centering
    \begin{subfigure}[b]{0.48\textwidth}
        \centering
        \includegraphics[width = \textwidth]{imagenes/logo.png}
        \caption{}
        \label{subfig:test4}
    \end{subfigure}
    \hfill
    \begin{subfigure}[b]{0.49\textwidth}
        \centering
        \includegraphics[width = \textwidth]{imagenes/logoITBA.png}
        \caption{}
        \label{subfig:test5}
    \end{subfigure}
    \hfill
    \begin{subfigure}[b]{\textwidth}
        \centering
        \includegraphics[width = \textwidth]{imagenes/logoITBA.png}
        \caption{}
        \label{subfig:test6}
    \end{subfigure}
        \caption{Ejemplo de subfiguras}
        \label{fig:test2}
\end{figure}

\negrita{\subrayar{TABLAS}}
\begin{table}[H]
  \centering
  \resizebox{7cm}{!}{
    \begin{tabular}{|c|c|c|c|c|c|c|}
    \hline
    a  & a & a & a & a & a\\
    \hline
    a  & a & a & a & a & a\\
    \hline
    a  & a & a & a & a & a\\
    \hline
    \end{tabular}
    }
    \caption{Ejemplo}
  \label{tab:test}
\end{table}

\negrita{\subrayar{FRAGMENTO DE CÓDIGO}}
\begin{lstlisting}[language=Python,caption={Ejemplo}]
Hola mundo
test = 0
if test:
	print("Se paso el test")
else:
	print("Hubo una falla")
 
lista = [[56, 34, 1],
         [12, 4, 5],
         [9, 4, 3]]
for i in lista:
    for j in i:
        print(j)
\end{lstlisting}

\negrita{\subrayar{ECUACIONES}}
\begin{equation}
    2 + 2 = 5
     \label{eq:test}
\end{equation}

\negrita{\subrayar{ECUACIONES DE REFERENCIA}}
\begin{center}
%https://manualdelatex.com/simbolos
\subrayar{Producto}
\begin{equation*}
	a\cdot b
\end{equation*}

\subrayar{División}
\begin{equation*}
	\frac{a}{b}
\end{equation*}

\subrayar{Integral}
\begin{align*}
	&\int f(x)\,dx\\
        &\iint f(x,y)\,dx\,dy\\
        &\iiint f(x,y,z)\,dx\,dy\,dz\\
        &\int\limits_{a}^{b}f(x)\,dx\\
        &\int_{a}^{b}f(x)\,dx\\
        &\int_{a}^{b}\int_{c}^{d}f(x,y)\,dx\,dy
\end{align*}

\subrayar{Sumatoria}
\begin{equation*}
	\sum_{i=1}^n f(x_i) \Delta x
\end{equation*}

\subrayar{Productoria}
\begin{equation*}
	\prod_{i=1}^n f(x_i) \Delta x
\end{equation*}

\subrayar{Límites}
\begin{align*}
	&\lim_{x \to 0} \frac{1}{x} = \,\infty \\ 
        &\frac{1}{x}\stackrel[] {x \to 0}{\longrightarrow}\,\infty
\end{align*}

\subrayar{Módulo}
\begin{equation*}
	\abs{-5}=5
\end{equation*}

\subrayar{Ángulos}
\begin{equation*}
	180\degree \, 90\grados
\end{equation*}

\subrayar{Raíces}
\begin{align*}
	&\sqrt{x+1}\\
        &\sqrt[3]{x+1}\\
        &\sqrt[4]{x+1}
\end{align*}

\subrayar{Matrices y vectores}
\begin{align*}\centering
        &    A_{m,n} = 
            \begin{pmatrix}
                a_{1,1} & a_{1,2} & \cdots & a_{1,n} \\
                a_{2,1} & a_{2,2} & \cdots & a_{2,n} \\
                \vdots  & \vdots  & \ddots & \vdots  \\
                a_{m,1} & a_{m,2} & \cdots & a_{m,n} 
            \end{pmatrix}\\
        &\vec{a} =
            \begin{pmatrix} 
                1\\2\\3 
            \end{pmatrix}\\
        &\widehat{\vec{a}}\\
        &\overrightarrow{AB}
\end{align*}

\subrayar{Sistemas}
\begin{equation*}
	f(x) = \begin{cases}
	x^2 & \text{si } x>2,\\
	x-1 & \text{si } x\leq 2
	\end{cases}
\end{equation*}

\subrayar{Simbolos}
\begin{equation*}
\pm \, \infty \, \leq \, \geq \, \circ \, \in \, \notin \, \subset \, \subseteq \, \cap \, \cup \, A^\complement \, \exists \, \forall \, \neq \, \bullet  \, \times \, \angle \, \land \, \lor \, \neg \, \overline{AB} \, \triangle \, \square
\end{equation*}

\subrayar{Flechas}
\begin{equation*}
\leftarrow \, \rightarrow \, \leftrightarrow \, \Leftarrow \, \Rightarrow \, \Leftrightarrow \, \uparrow \, \downarrow \, \Updownarrow
\end{equation*}

\subrayar{Funciones}
\begin{equation*}
	\sin(x) \quad \cos(x) \quad \tan(x) \quad \ln(x) \quad \log(x) \quad \exp(x)
\end{equation*}

\subrayar{Entorno \cursiva{align}}
%se usa el & para alinear los distintos renglones
\begin{align*}
	(a+b)^2 &= (a+b)(a+b)\\
	&= a^2+ab+ba+b^2\\
	&=a^2+2ab+b^2
\end{align*}

\subrayar{Distintos tipos de letras}
\begin{equation*}
    \mathrm{R}, \mathbb{R}, \mathcal{R}, \mathfrak{R}, \mathbf{R}, \mathsf{R}, \mathit{R}, \mathscr{R}
\end{equation*}

\subrayar{\large{Ejemplos}}
\begin{equation*}
\mathscr{F}[x(t)](f)=X(f)=\int_{-\infty}^{\infty} x(t) \supind{e}{-i 2\pi f t}\, dt
\end{equation*}

\begin{equation*}
\mathscr{F}[x(n)](f)=X(f)=\sum_{-\infty}^{\infty} x(n) \supind{e}{-i 2\pi f n}
\end{equation*}

\begin{equation*}
I = \iint{\overrightarrow{J}\cdot\overrightarrow{ds}}=J\cdot A
\end{equation*}

\begin{equation*}
\subind{V}{esfera} = \iiint_{Bola}\,dV = \int_{0}^{2\pi}\int_{0}^{\pi}\int_{0}^{R}{r^2\sin(\phi)}\,dr\,d\phi\,d\theta=\frac{4}{3}\pi R^2
\end{equation*}

\begin{equation*}
f'(x)=\lim_{\Delta x \to 0} \frac{f(x+\Delta x)-f(x)}{\Delta x}
\end{equation*}

\begin{equation*}
	sgn(x) = \begin{cases}
	-1 & \text{si } x<0,\\
	0 & \text{si } x=0,\\
        1 & \text{si } x>0
	\end{cases}
\end{equation*}

\begin{equation*}
	\frac{d^2}{dt^2}i(t)+\frac{R}{L}\frac{d}{dt}i(t)+\frac{1}{LC}i(t)=0
\end{equation*}

\begin{equation*}
    \mathbf{X} = 
            \begin{pmatrix}
                1&2\\
                3&4\\
                5&6
            \end{pmatrix}\in \supind{\mathbb{R}}{3\times 2}
\end{equation*}
\end{center}

\negrita{\subrayar{FÓRMULAS QUÍMICAS}}
\begin{center}
\chemfig{A-B}\\
\chemfig{A=B}\\
\chemfig{A~B}\\
\chemfig{A>B}\\
\chemfig{A<B}\\
\chemfig{A>:B}\\
\chemfig{A<:B}\\
\chemfig{A>|B}\\
\chemfig{A<|B}\\

\chemfig{(-[:0,1.5,,,draw=none]\scriptstyle\color{red}0) (-[1]1)(-[:45,1.5,,,draw=none]\scriptstyle\color{red}45) (-[2]2)(-[:90,1.5,,,draw=none]\scriptstyle\color{red}90) (-[3]3)(-[:135,1.5,,,draw=none]\scriptstyle\color{red}135) (-[4]4)(-[:180,1.5,,,draw=none]\scriptstyle\color{red}180) (-[5]5)(-[:225,1.5,,,draw=none]\scriptstyle\color{red}225) (-[6]6)(-[:270,1.5,,,draw=none]\scriptstyle\color{red}270) (-[7]7)(-[:315,1.5,,,draw=none]\scriptstyle\color{red}315) -0}

\chemfig{A*6(-B-C-D-E-F-)}

2-etil-5-propilnon-1,5-dien-3-ino

\chemfig{C=C(-[6]C(-[6]C))-C~C-C(=[6]C(-[6]C(-C-C)))-C-C-C}

\end{center}
