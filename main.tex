% informe.tex
% Copyright (C) 2024 Cristian Meichtry
% https://github.com/cmeichtry/template_LATEX

\documentclass[11pt]{article}
% preamble.tex
% Copyright (C) 2023 Cristian Meichtry
% https://github.com/cmeichtry/template_LATEX
%--------------------BLOQUE DE FORMATO--------------------%
\usepackage[a4paper,includehead,includefoot, left=2cm,top=1.2cm,right=2cm,bottom=1.5cm,headheight=25pt]{geometry} %margenes.
%--------------------FIN DE BLOQUE DE FORMATO--------------------%

%--------------------CONFIGURACION DE IDIOMA--------------------%
\usepackage[spanish,es-tabla]{babel} % pone el idioma en español, es-tabla para Tabla en lugar de Cuadro
\usepackage[utf8]{inputenc} % Codificación de entrada, carácteres acentuados, ñ
\usepackage[T1]{fontenc} % Codificación de fuente, habilita caracteres no latinos
\usepackage{lmodern} % Fuente compatible
%--------------------FIN DE CONFIGURACION DE IDIOMA--------------------%

\usepackage{amsmath,amsthm,amsfonts,amssymb,amscd}  %para ecuaciones
\usepackage{enumerate} %permite hacer listas numeradas y por niveles
\usepackage{mathrsfs} %letras mayúsculas especiales, cursiva, gotica, etc
\usepackage{dsfont} %letras mayúsculas para conjuntos (Ej: Naturales)
\usepackage{xcolor} %permite escribir texto en color
\usepackage{graphicx} %para insertar graficos e imagenes.
\usepackage{listings} %para ingresar codigos de MATLAB, Python, Fortran, etc
\usepackage{bm} %permite usar simbolos matematicos en negrita
\usepackage{nicefrac} %Permite escribir fracciones en el medio del texto de una forma elegante \nicefrac{a}{b}
\usepackage{booktabs} %modificación de las lineas en las tablas
\usepackage{parskip} %modificación de sangría
\usepackage[center]{caption} %permite modificar el formato de los caption para tablas/figuras
\usepackage{array} %permite escribir en columnas sin ser una tabla, por ejemplo, matrices
\usepackage{multirow} %filas múltiples en tablas
\usepackage{stackrel} %Permite escribir cosas arriba o abajo de otras, \stackrel[lo de abajo] {lo de arriba}{lo del medio}
\usepackage{subfigure} %Permite poner subfiguras
\usepackage{cancel} %permite tachar terminos en ecuaciones
\usepackage{comment} %para comentar grandes partes de codigo, usar \begin{comment}
\usepackage{setspace} %permite cambiar espacio entre lineas
\usepackage{steinmetz} %mejora la forma de escribir numeros complejos
\usepackage{fancyhdr} % Modificar encabezados y pies de paginas.
\usepackage{titlesec} %modifición de secciones
\usepackage{float} %para el float H en tablas.
\usepackage{etoolbox} %crea condicionales, puedo tener dos PDFs distintos desde un solo archivo fuente
\usepackage[colorlinks=false,linkcolor=black]{hyperref} %pone vinculos en el indice y en las referencias
\usepackage{lipsum} %introduce bloques de texto random, usado para tests
%--------------------Titulos y enumeración de secciones--------------------%
%titulos de secciones y subsecciones.
\titleformat{\section}[hang]{\normalfont\bfseries\centering}{\thesection}{5pt.}{} %modificación secciones.
\titleformat{\subsection}[hang]{\normalfont\bfseries\raggedright}{\thesubsection}{5pt}{} %modificación subsecciones.
\titleformat{\subsubsection}[hang]{\normalfont\itshape}{\thesubsubsection}{5pt}{} %modificación subsecciones.
%\titleformat{\subsubsection}[hang]{\itshape}{Section \thesubsubsection}{}{\hspace{1cm} $\bullet$ \hspace{0.1cm}}[] %modificación subsecciones.

%enumeración de secciones y subsecciones.
\renewcommand{\labelenumi}{\arabic{section}.\arabic{enumi}}
\renewcommand{\labelenumii}{\arabic{section}.\arabic{enumi}.\arabic{enumii}}
\renewcommand{\labelenumiii}{\arabic{section}.\arabic{enumi}.\arabic{enumii}.\arabic{enumiii}}
\renewcommand{\labelenumiv}{\arabic{section}.\arabic{enumi}.\arabic{enumii}.\arabic{enumiii}.\arabic{enumiv}}

\renewcommand{\thefigure}{\arabic{section}.\arabic{figure}}  %definición enumeración de ecuaciones.
\renewcommand{\thetable}{\arabic{section}.\arabic{table}} %definición enumeración de tablas.

\setcounter{tocdepth}{2} %hasta que nivel aparece en el índice
\setcounter{secnumdepth}{3} %hasta que nivel se enumera
%niveles 1:section, 2:subsection, 3:subsubsection...

\setlength{\parindent}{12pt} %sangria
\linespread{1} % Interlineado

%---------------- ENCABEZADO Y PIE DE PAGINA ----------------%
\pagestyle{fancy}
\fancyhf{}
\rhead{\begin{picture}(0,0) \put(-51,-7){\includegraphics[width=20mm]{\logofac}} \end{picture}} %incluye logo en esq sup derecha
\lhead{\materia} %incluye nombre de materia en esq sup izquierda
%\lfoot{\leftmark} %incluye nombre de subsección en esq inf izquierda
\rfoot{\thepage} %incluye número de página en esq inf derecha
\renewcommand{\headrulewidth}{1pt} %línea para margen superior
%\renewcommand{\footrulewidth}{1pt} %línea para margen inferior

%---------------- INICIO DEL DOCUMENTO ----------------%
\begin{document}

%------------------ DATOS GENERALES ------------------%
\def \universidad{FACULTAD}
\def \unidad{Trabajo Práctico N$^\circ$X}
\def \carrera{CARRERA O DEPARTAMENTO}
\def \materia{MATERIA}
\def \titulo{TITULO}
\def \logofac{preambulo/logo.png} %cambiar el nombre del archivo segun la imagen que corresponda

%El número de grupo y el nombre de autores, ayudantes, y docentes se modifican sobre la cáratula en el .tex correspondiente 

%---------------- CARATULA ----------------%
% Modelo de caratula 1
% Copyright (C) 2023 Cristian Meichtry
% https://github.com/cmeichtry/template_LATEX

\begin{titlepage}
    \begin{center}
        \includegraphics[width=0.4\linewidth]{\logofac}\\
         \vspace{-0.5cm}
        \textbf{\huge{\universidad}}
        
        \Large{\carrera}
        
         \vspace{-0.3cm}
        \LARGE{\materia}
        
        \vspace{-0.8cm}
        \rule{10cm}{0.01cm}\\
        \LARGE{\textbf{\unidad}}
        
        \vspace{-0.3cm}
        
        \Large{\titulo}
        
     \vspace{-0.3cm}
        \normalsize
        \rule{10cm}{0.01cm}\\
        \vspace{0.3cm}
        \autores
        \vspace{0.1cm}
        \textbf{PROFESORES}\\
        \profesores
        %\vspace{0.4cm}
        %\textbf{AYUDANTES}\\ %descomentar en caso de agregar a los ayudantes
        %\Colaboradores
        \rule{10cm}{0.01cm}\\
        
        
\end{center}
    \noindent
    \hspace{30pt}Fecha de entrega: \today ~~~~~~~~~~~~~~~~~~~~~~~~~~~~~ NOTA:
    
    \noindent
    \hspace{30pt}Fecha de corrección:
    
    \noindent
    \hspace{30pt}Observaciones: 
        
    
\end{titlepage} %elegir modelo de caratula
\newpage

%---------------- INDICE ----------------%
\newpage
\pagestyle{empty}
\renewcommand{\contentsname}{{\textit{\huge Índice}}} %Se puede poner lo que sea en vez de índice
\tableofcontents

%---------------- LISTA DE FÍGURAS ----------------%
\newpage
\pagestyle{empty}
\renewcommand{\listfigurename}{{\textit{\huge Lista de figuras}}}
\listoffigures

%---------------- DESARROLLO ----------------%
\newpage
\pagestyle{fancy} %activa encabezado y pie de pagina
\setcounter{page}{1} %empezar numeración desde está pagina
\seteocontadores
\section{Sección 1}
\label{sec:seccion1}
\subfile{secciones/seccion1}

\newpage
\seteocontadores
\section{Sección 2}
\label{sec:seccion2}
\subfile{secciones/seccion2}

\newpage
\seteocontadores
\section{Sección 3}
\label{sec:seccion3}
\subfile{secciones/seccion3}

\newpage
\seteocontadores
\section{Sección 4}
\label{sec:seccion4}
\subfile{secciones/seccion4}

\newpage
\seteocontadores
\section{Sección 5}
\label{sec:seccion5}
\subfile{secciones/seccion5}

\newpage
\seteocontadores
\section{Sección 6}
\label{sec:seccion6}
\subfile{secciones/seccion6}

%%%%%%% APENDICES %%%%%%%%
\appendix %define el formato apendice
\clearpage
\newpage
\pagenumbering{Roman} %cambia numeracion
\def \nombreapendice{A} %letra del apendice para tablas y figuras
\formatoapendice %define el formato para numeracion de cosas
\seteocontadores %inicializa los contadores
\section{Nombre Apéndice A}\label{ApendiceA} %nombre del apendice, mas label para referenciarlo (se puede con \autoref{})
\subfile{secciones/apendiceA}

\clearpage
\newpage
\pagenumbering{Roman}
\def \nombreapendice{B}
\formatoapendice
\seteocontadores
\section{Nombre Apéndice B}\label{Apendice B}
\subfile{secciones/apendiceB}

%---------------- BIBLIOGRAFIA ----------------%
\clearpage
\newpage
\pagenumbering{Roman}
\seteocontadores
\rfoot{ }
\lfoot{ }
\input{secciones/bibliografia.bib} %agrega la bibliografía

%agrega la sección bibliografía en el indice

%---------------- FIN DEL DOCUMENTO ----------------%
\end{document} 
